\documentclass{jsreport}
\usepackage[dvipdfmx]{graphicx}
\usepackage{comment}
\title{アーキテクチャ検討報告書} 
% アーキテクチャ検討報告書 (ファイル名:architecture_study_report.pdf) 「何を目指すか」
\author{グループ番号:20 \\学生番号1029285143 原口卓也
\\学生番号1029287621 宮城竜大} 
\date{\today}
\begin{document}
\begin{titlepage}
\maketitle
\end{titlepage} 
\clearpage 
\section{設計目標} 
今回、私達はソートを実行する際の動作速度の上昇と必要サイクル数の短縮を目指し、プロセッサを設計する。
5段パイプライン化を最大の目標とし、比較的実装が容易で、かつ、サイクル数の短縮に役立ちそうな、即値演算命令(ADDI、SUBI)と分岐命令(BR、BAL)の実装を行う。
今回は、あくまで機能の実現を目標とした。

% 実現する機能、性能の目標数値、など
%%%%%%%%%%%%%%%%%%%%%%%%%%%%%%%%%%%%%%%%%%%%%%%%%%%%%%%%%%%%%%%%%%%%%%%%
% todo
%%%%%%%%%%%%%%%%%%%%%%%%%%%%%%%%%%%%%%%%%%%%%%%%%%%%%%%%%%%%%%%%%%%%%%%% 

\section{高速化/並列処理の方式}
% 拡張命令、動作周波数、パイプライン化、並列化、など 	
\subsection{即値演算命令の追加}       
SIMPLEのアーキテクチャでは、演算命令のオペランドはいずれも汎用レジスタであるため、定数との演算の際、定数をレジスタに読み込む命令と、実際に演算する命令の2つの命令が必要となっていた。
ここにレジスタRdとdフィールドで指定した即値を演算する命令を追加し、命令サイズの縮小を図る。    
\subsection{Branch Register, Branch And Link命令の追加}    
SIMPLEのアーキテクチャでは、レジスタの戻り先を格納していないため、関数呼び出しのように、任意の処理任意の箇所で行ったあと戻るといった操作ができなかった。
そこでBranch Register, Branch And Link命令を追加し、関数呼び出しを可能にする。        
\subsection{パイプライン化}     	
p1/p2/p3/p4/p5の並列実行を可能にしパイプライン化する。 

\section{性能/コストの予測} 
パイプライン化を行うため、サイクル数が大幅に短縮されることが予想される。

% SIMPLE/Bに比べて性能やハードウェア量が何倍程度か、ソート速度コンテストでの計算時間、サイクル数の予測、など
%%%%%%%%%%%%%%%%%%%%%%%%%%%%%%%%%%%%%%%%%%%%%%%%%%%%%%%%%%%%%%%%%%%%%%%%
% todo
%%%%%%%%%%%%%%%%%%%%%%%%%%%%%%%%%%%%%%%%%%%%%%%%%%%%%%%%%%%%%%%%%%%%%%%%   

\section{考察等} 
%%%%%%%%%%%%%%%%%%%%%%%%%%%%%%%%%%%%%%%%%%%%%%%%%%%%%%%%%%%%%%%%%%%%%%%%
% todo
%%%%%%%%%%%%%%%%%%%%%%%%%%%%%%%%%%%%%%%%%%%%%%%%%%%%%%%%%%%%%%%%%%%%%%%% 

\end{document}

