\documentclass{jsreport}
\usepackage[dvipdfmx]{graphicx}
\usepackage{comment}
\title{アーキテクチャ検討報告書}
% アーキテクチャ検討報告書 (ファイル名:architecture_study_report.pdf) 「何を目指すか」

\author{学生番号1029285143 原口卓也 \and 学生番号1029287621 宮城竜大}
\date{\today}

\begin{document}

\begin{titlepage}
\maketitle
\end{titlepage}
\clearpage

\section{要求仕様、設計目標、方針、特長}
% 実現する機能、性能の目標数値、など
去年のタイムを参考に、最小限の機能実装で、ランキングの真ん中あたりに乗れるようなタイム(約2.000ms)あたりを目指す。
そのためにパイプライン実装は必須と考えられる。

\section{高速化/並列処理の方式}
% 拡張命令、動作周波数、パイプライン化、並列化、など
	\subsection{即値演算命令の追加}
        SIMPLEのアーキテクチャでは、演算命令のオペランドはいずれも汎用レジスタであるため、定数との演算の際、定数をレジスタに読み込む命令と、実際に演算する命令の2つの命令が必要となっていた。ここにレジスタRdとdフィールドで指定した即値を演算する命令を追加し、命令サイズの縮小を図る。
    
    \subsection{Branch Register, Branch And Link命令の追加}
    	SIMPLEのアーキテクチャでは、レジスタの戻り先を格納していないため、関数呼び出しのように、任意の処理任意の箇所で行ったあと戻るといった操作ができなかった。そこでBranch Register, Branch And Link命令を追加し、関数呼び出しを可能にする。
        
	\subsection{パイプライン化}
    	p1/p2/p3/p4/p5の並列実行を可能にしパイプライン化する。

\section{性能/コストの予測}
% SIMPLE/Bに比べて性能やハードウェア量が何倍程度か、ソート速度コンテストでの計算時間、サイクル数の予測、など
	\subsection{即値演算命令の追加}
		即値演算命令の命令サイズが半分になるため、該当の命令数が多ければ性能に大きく影響する。\\
        ハードウェアサイズもALUの拡張で済むためあまり大きくならないと思われる。
        
    \subsection{Branch Register, Branch And Link命令の追加}
    	関数の呼び出しは、命令数を減らすことができるが、実際の実行命令数はかわらないため、速度には影響しない。また、PC周りに少量の追加で済む。
        
	\subsection{パイプライン化}
    	単純計算で速度が5倍程度になる。
        
\end{document}
