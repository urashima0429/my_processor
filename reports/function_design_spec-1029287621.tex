\documentclass{jsreport}
\usepackage[dvipdfmx]{graphicx}
\usepackage{comment}
\title{機能設計仕様書}
% 機能設計仕様書 (ファイル名:function_design_spec-[学籍番号].pdf) 「具体的な実現方法」

\begin{comment}

    機能設計では, 方式設計によって定められたシステムを実現するため, 全体をいくつかの機能コンポーネントに分割し, 各コンポーネントの構造 (structure) と動作 (behavior) を定める.

    計算機の場合には, 以下のようなコンポーネントに分割するのが適当であろう:

        制御回路
        PC 周辺
        レジスタ・ファイル周辺
        演算器 (AU, LU, シフタ)
        バス・インターフェイス 

    フェーズごとに分割するのは, 同一の『もの』が異なる複数のフェーズで動作することになるため, うまく行かない.

    機能設計レベルの仕様書は, それがあれば, 誰がその先の設計, すなわち論理設計を行っても, そのコンポーネントの動作周波数とチップ面積に 大きな差が出ないように記述されていなければならない. 具体的には, およそ以下の項目が含まれる:

    外部仕様

            特徴
            方式設計, 他のコンポーネントとの関わり: コンポーネントの, 計算機全体における位置付け, 役割. 外部とのインターフェイス, 外部から見た動作.

            構造
                計算機全体のブロック図における, そのコンポーネントの位置づけ. 
            動作
                計算機全体のフェーズ・フロー・チャート, データ・フロー・チャートにおける, そのコンポーネントの位置づけ. 

        他のコンポーネントを担当者は, このコンポーネントの外部仕様だけを参照すればよい.
    内部仕様

            実装上の特徴
            コンポーネントの内部構造と動作:

            構造
                コンポーネントのブロック図 (論理回路図ではない), またはアルゴリズム (プログラムではない). 
            動作
                コンポーネントのフェーズ・フロー・チャート, データ・フロー・チャート, タイミング・チャート. 

            論理設計にあたって特に留意すべき点. 例えば, critical path の指摘など. 
\end{comment}

\author{学生番号1029287621 宮城竜大}
\date{\today}
\begin{document}

\maketitle
\clearpage

\section{コンポーネントの分割について}

以下のように大まかに分割している(担当)
\begin{itemize}
\item 制御回路(宮城)
\item PC周辺(宮城)
\item レジスタ・ファイル周辺(原口)
\item 演算器 (AU, LU, シフタ)(原口) 
\item バス・インターフェイス(原口)
\end{itemize}

以下に制御回路とPC周辺の機能設計仕様書を示す。

\section{制御回路}
	\subsection{外部仕様}
        \subsubsection{特徴}
        	SIMPLEアーキテクチャで用いたものと同じく、プロセッサー全体の制御を行う。入力exec, reset信号を受け取り、各レジスタに的確な信号を送ることで、フェーズの管理、実行、停止、初期化を行うユニット。
            
        \subsubsection{構造}
        	ユーザーからのexec,reset入力に対して、フェーズ信号出力の停止・再開、レジスタの初期化などを行う。また、関数呼び出し時など、ユーザー入力だけでなく自身からのreset入力にたいしてもレジスタの初期化を行う。
            \begin{itemize}
            \item 入力:clock(1bit), exec(1bit), reset(1bit)
            \item 出力:register\_clock(1bit), register\_reset(1bit)
            \end{itemize}

        \subsubsection{動作}
        	内部変数running が正の時
            \begin{itemize}
        		\item clockの立ち上がりに同期してregister\_clockを立ち上げる
            	\item execの立ち上がりに同期して内部変数runningを負にする
                \item resetの立ち上がりに同期してregister\_reset信号を立ち上げる
            \end{itemize}
            
            内部変数running が負の時
            \begin{itemize}
            	\item execの立ち上がりに同期して内部変数runningを正にする
                \item resetの立ち上がりに同期してregister\_reset信号を立ち上げる
            \end{itemize}

    \subsection{内部仕様}
        \subsubsection{特徴}
			SIMPLEアーキテクチャで用いたものと同じく、プロセッサー全体の制御を行う。入力exec, reset信号を受け取り、各レジスタに的確な信号を送ることで、フェーズの管理、実行、停止、初期化を行うユニット。パイプライン化するため、フェーズの管理が必要なくなり、より簡潔となっている。
            
        \subsubsection{構造}
        	単一ユニットで作成するため、ブロック図は省略
            
        \subsubsection{動作}
        	パイプライン化するため、任意のフェーズにおいて、以下の同一の動作をする。\\
            内部変数running が正の時
            \begin{itemize}
        		\item clockの立ち上がりに同期してregister\_clockを立ち上げる
            	\item execの立ち上がりに同期して内部変数runningを負にする
                \item resetの立ち上がりに同期してregister\_reset信号を立ち上げる
            \end{itemize}
            
            内部変数running が負の時
            \begin{itemize}
            	\item execの立ち上がりに同期して内部変数runningを正にする
                \item resetの立ち上がりに同期してregister\_reset信号を立ち上げる
            \end{itemize}

\section{PC周辺}
	\subsection{外部仕様}
        \subsubsection{特徴}
        	SIMPLEアーキテクチャで用いたものと同じく、実行中の命令のメモリアドレスを保持する。
            
            clock入力に同期してPCを更新する。その際、\\
            \begin{itemize}
            \item 分岐命令の際は、該当のアドレスへ更新する。
            \item 関数呼び出しの際は、現在のアドレスをレジスタに格納してから、該当のアドレスへ更新する。
            \item 関数戻りの際は、指定レジスタのアドレスへ更新する。
			\item それら以外の際は、PC + 1へ更新する。
            \end{itemize}
        
        \subsubsection{構造}
        
            入力
            \begin{itemize}
            \item clock(1bit)
            \item reset(1bit)
            \item data0(16bit)(=adder通過後のPC+1の値)
            \item data1(16bit)(=MDRからの戻り値)
            \item order1(1bit)(=switcherからの分岐命令時の値)
            \item order2(1bit)(=switcherから関数呼び出し時の値)
            \item order3(1bit)(=switcherから関数戻り時の値)
            \end{itemize}
            
            出力
            \begin{itemize}
            \item out(PC)(16bit)
            \item out(adder)(16bit)
            \end{itemize}
            
        \subsubsection{動作}
        clock入力に同期してPCのメモリアドレスを更新する。
        \begin{itemize}
        \item order1が正の時 PC = data1
		\item order2が正の時 Rd = PC + 1 , PC = data1
        \item order3が正の時 PC = Rd
        \item それ以外の時    PC = PC + 1
    	\end{itemize}
        
    \subsection{内部仕様}

        \subsubsection{特徴}
        SIMPLEアーキテクチャで用いたものと同じく、実行中の命令のメモリアドレスを保持する。                
            clock入力に同期してPCを更新する。その際、\\
            \begin{itemize}
            \item 分岐命令の際は、該当のアドレスへ更新する。
            \item 関数呼び出しの際は、現在のアドレスをレジスタに格納してから、該当のアドレスへ更新する。
            \item 関数戻りの際は、指定レジスタのアドレスへ更新する。
			\item それら以外の際は、PC + 1へ更新する。
            \end{itemize}
        
        \subsubsection{構造}
        
        
        \subsubsection{動作}    
        clock入力に同期してPCのメモリアドレスを更新する。
        \begin{itemize}
        \item order1が正の時 PC = data1
		\item order2が正の時 Rd = PC + 1 , PC = data1
        \item order3が正の時 PC = Rd
        \item それ以外の時    PC = PC + 1
    	\end{itemize}
        
\end{document}
