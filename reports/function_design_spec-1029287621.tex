\documentclass{jsreport}
\usepackage[dvipdfmx]{graphicx}
\usepackage{comment}
\title{機能設計仕様書}
% 機能設計仕様書 (ファイル名:function_design_spec-[学籍番号].pdf) 「具体的な実現方法」

\begin{comment}

    機能設計では, 方式設計によって定められたシステムを実現するため, 全体をいくつかの機能コンポーネントに分割し, 各コンポーネントの構造 (structure) と動作 (behavior) を定める.

    計算機の場合には, 以下のようなコンポーネントに分割するのが適当であろう:

        制御回路
        PC 周辺
        レジスタ・ファイル周辺
        演算器 (AU, LU, シフタ)
        バス・インターフェイス 

    フェーズごとに分割するのは, 同一の『もの』が異なる複数のフェーズで動作することになるため, うまく行かない.

    機能設計レベルの仕様書は, それがあれば, 誰がその先の設計, すなわち論理設計を行っても, そのコンポーネントの動作周波数とチップ面積に 大きな差が出ないように記述されていなければならない. 具体的には, およそ以下の項目が含まれる:

    外部仕様

            特徴
            方式設計, 他のコンポーネントとの関わり: コンポーネントの, 計算機全体における位置付け, 役割. 外部とのインターフェイス, 外部から見た動作.

            構造
                計算機全体のブロック図における, そのコンポーネントの位置づけ. 
            動作
                計算機全体のフェーズ・フロー・チャート, データ・フロー・チャートにおける, そのコンポーネントの位置づけ. 

        他のコンポーネントを担当者は, このコンポーネントの外部仕様だけを参照すればよい.
    内部仕様

            実装上の特徴
            コンポーネントの内部構造と動作:

            構造
                コンポーネントのブロック図 (論理回路図ではない), またはアルゴリズム (プログラムではない). 
            動作
                コンポーネントのフェーズ・フロー・チャート, データ・フロー・チャート, タイミング・チャート. 

            論理設計にあたって特に留意すべき点. 例えば, critical path の指摘など. 
\end{comment}

\author{学生番号1029287621 宮城竜大}
\date{\today}
\begin{document}

\maketitle
\clearpage

\begin{figure}[htbp]
  \begin{center}
    \includegraphics[width=60mm]{figures/Sample.png}
    \caption{コンポーネント分割図}
  \end{center}
\end{figure}


\section{コンポーネント1}
	\subsection{外部仕様}
        \subsubsection{特徴}
        \subsubsection{構造}
        \subsubsection{動作}

    \subsection{内部仕様}
        \subsubsection{特徴}
        \subsubsection{構造}
        \subsubsection{動作}
        
\section{コンポーネント1}
	\subsection{外部仕様}
        \subsubsection{特徴}
        \subsubsection{構造}
        \subsubsection{動作}    
    
    \subsection{内部仕様}
        \subsubsection{特徴}
        \subsubsection{構造}
        \subsubsection{動作}    
\end{document}
